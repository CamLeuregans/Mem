\documentclass[a4paper,12pt,twoside,final]{book} 	% Feuille A4, police principale à 12pt, recto-verso


%%%%%%%%%%%%%%%%%% Définitions variables globales %%%%%%%%%%%%%%%%%%

\def\TitreMemoire{La technologie dans le cadre de l\textquoteright entrainement de natation}
\def\AuteurMemoire{Camille Leuregans}
\def\MaitreApp{Rahma Boudouda}
\def\TuteurEns{Jean-Fran\c cois Pradat-Peyre}


%%%%%%%%%%%%%%%%%%%%%%%%%%%%% Packages %%%%%%%%%%%%%%%%%%%%%%%%%%%%%

\PassOptionsToPackage{english, francais}{babel}			% Langues
\usepackage[utf8]{inputenc}								% Accents (encodage en utf8)
\usepackage[T1]{fontenc}									% Caractères français
\usepackage{lmodern}
\usepackage{graphicx}									% Images
\usepackage{setspace}
\usepackage{fancyhdr}									% Pour les entêtes et pieds de page
\usepackage{hyperref}									% Pour les références
\usepackage[french]{varioref}							% De la table des matières
\usepackage[hmargin=2cm, vmargin=2cm]{geometry}			% Marges
\usepackage[dvipsnames, svgnames]{xcolor}
\usepackage{shorttoc}									% Pour le sommaire
\usepackage{glossaries}									% Pour le glossaire
\makeglossaries
\usepackage{nomencl}										% Pour la table des sigles et abréviations
\makenomenclature
%\renewcommand{\nomname}{Table des sigles et abréviations}
\usepackage[nottoc, notlof, notlot]{tocbibind}			% Pour mettre la bibliograhie dans la table des matières
\usepackage{wallpaper}									% Ajout du fond de la couverture
\usepackage[frenchb]{babel}


%%%%%%%%%%%%%%%%%%%%%%%%%%% Définitions %%%%%%%%%%%%%%%%%%%%%%%%%%%

\setcounter{secnumdepth}{3}								% On défini le nombres de sous chapitres à identifier à 3

\renewcommand{\thesection}{\Roman{section}}				% Chiffres romain des chapitres
\renewcommand{\thesubsubsection}{\alph{subsubsection}}	% Lettres minuscules pour les sous sous chapitres


\pagestyle{fancy}										% Pour les entêtes et pieds de page

\singlespacing


%%%%%%%%%%%%%%%%%%%%%%%% Des abréviations %%%%%%%%%%%%%%%%%%%%%%%%

\nomenclature{abdos}{abdominaux}
\nomenclature{CPB}{Cercle Paul Bert}
\nomenclature{h}{heures}
\nomenclature{kcal}{kilocalories}
\nomenclature{kiné}{kinésithérapeute}
\nomenclature{km}{kilomètres}
\nomenclature{ostéo}{ostéopathe}
\nomenclature{m}{mètres}
\nomenclature{min}{minute}
\nomenclature{muscu}{musculation}
\nomenclature{ppg}{préparation physique générale}
\nomenclature{récup}{récupération}
\nomenclature{sec}{secondes}
\nomenclature{spé}{spécifique}
\nomenclature{x}{fois}
\nomenclature{4n}{4 nages}

%%%%%%%%%%%%%%%%%% Liens et couleurs du document %%%%%%%%%%%%%%%%%%

\hypersetup{
colorlinks=true,
linkcolor=Navy,
filecolor=Navy,
urlcolor=Navy
}

\xdefinecolor{emeraude}{named}{Emerald}
\xdefinecolor{marine}{named}{Navy}
\definecolor{lightgray}{rgb}{0.9,0.9,0.9}
\definecolor{darkgray}{rgb}{0.4,0.4,0.4}
\definecolor{purple}{rgb}{0.65, 0.12, 0.82}


%%%%%%%%%%%%%%%%%%% En-têtes et pieds de pages %%%%%%%%%%%%%%%%%%%

\fancyfoot{}
\fancyfoot{}
\fancyhead[L]{\rightmark}
\fancyhead[C]{}
\fancyhead[R]{\AuteurMemoire}
\fancyfoot[L]{{M2 MIAGE App}}
\fancyfoot[C]{\thepage}
\fancyfoot[R]{Années 2016/2017}


%%%%%%%%%%%%%%%%%%%%%%%%%%% Document %%%%%%%%%%%%%%%%%%%%%%%%%%%

\begin{document}
  \begin{titlepage}
\LRCornerWallPaper{1}{img/fond.png}


		% Université
\begin{center}
\color{white}
\begin{LARGE}
\vspace*{-1.5cm}
\textbf{\textsf{Université Paris Nanterre}}
\end{LARGE}
\end{center}


		% Mémoire et année
\vspace{-0.8cm}
\begin{center}
\fontsize{38}{70}\selectfont
\textbf{\texttt{       Mémoire de fin d’études \\ }}
\vspace{0.8cm}
\fontsize{42}{70}\selectfont
\textbf{\texttt{       Master  MIAGE}}
\end{center}
\vspace*{3cm}
\begin{center}
\begin{huge}
\textbf{\textsf{       Année Universitaire 2016-2017}}
\end{huge}
\end{center}



		% Titre
\vspace*{0.2cm}
\begin{center}
\color{white}
\fontsize{30}{35}\selectfont
\textbf{{       \TitreMemoire}}
\end{center}



		% Auteur
\vspace*{0.4cm}
\begin{center}
\color{white}
\large
\hspace*{0.5cm}
\textbf{       Réalisé par Mme {\AuteurMemoire}}
\end{center}



		% Maitre App & Tuteur Ens
\vspace*{1.5cm}
\color{blue}
\begin{normalsize}
\textbf{Maitre d’apprentissage : \\ }
\color{black}
\hspace*{0.5cm}
{\MaitreApp \\}
\vspace*{1cm}

\color{blue}
\textbf{Tuteur Enseignant : \\ }
\color{black}
\hspace*{0.5cm}
{\TuteurEns}
\end{normalsize}




		%Logo
\vspace*{3cm}
\includegraphics[scale=1]{img/UPX.png}
\hspace*{3.5cm}
\includegraphics[scale=0.04]{img/Axa.png}
\hspace*{3.5cm}
\includegraphics[scale=1]{img/CFA.png}

\end{titlepage}
\ClearShipoutPicture

  \cleardoublepage
  \sloppy
  \section*{Remerciements}
\addcontentsline{toc}{section}{Remerciements} % Ajout dans la table des matières


TODO

\vspace{12pt}

Avant toutes choses, je tiens à remercier l'ensemble des personnes qui ont été là pour moi durant ce stage
et qui ont contribué au bon déroulement de celui-ci, mais aussi à la rédaction de ce rapport.


  \cleardoublepage
  \shorttableofcontents{Sommaire}{2}	% Le sommaire (profondeur 2)
  \addcontentsline{toc}{section}{Sommaire}	% Ajout dans la table des matières
  \cleardoublepage					% Ajout d'une page blanche si besoin (recto-verso)
  \section*{Introduction}
\addcontentsline{toc}{section}{Introduction}	% Ajout dans la table des matières


Dans le cadre de ma deuxième année de master MIAGE en apprentissage, j'ai pu intégrer l'entreprise AXA France, pour une durée d'un an, afin de compléter au mieux ma formation scolaire. Au cours de cette année, j'ai intégré le service "prévention risques activités opérationnelles et gestion de crise" au sein de la direction de la sureté, elle-même au sein du secrétariat général. Ma mission a pour descriptif "assistante chef de projet informatique" et consiste à prévenir/identifier des fraudes internes grâce à l'outil ACL, mais aussi participer à la formation des collaborateurs à l'outil ACL, animer des réunions, ect...

Lors de cette année d'apprentissage j'ai et je continue d'apprendre énormément, aussi bien à l'université que dans l'entreprise AXA France grâce à ma tutrice {\MaitreApp}. Cependant, suite aux conseils de mes enseignants, j'ai choisi de réaliser mon mémoire sur un sujet complètement différent de ce sur quoi j'ai pu travailler en entreprise.

\vspace{12pt}

Ce choix, s'est naturellement porté sur un sujet concernant l'entrainement sportif et même plus précisément l'entrainement en natation. J'ai réussi, depuis le collège, à concilier mes études avec ma pratique de la natation à haut niveau. Ma passion pour ce sport explique mon choix.

Comme mes études ne concernent en rien la natation, je vais m'intéresser à un outil utilisable dans le quotidien d'un nageur. Cet outil, totalcoaching.com, est une application multi-device qui peut être utilisé par des nageurs, mais aussi par n'importe quel autre sportif. Il permet de gérer les différents paramètres d'accès à la performance, que j'expliciterai au cours de ce mémoire, mais aussi d'éviter le surentrainement.

Lors de ma pratique sportive, j'aurai aimé avoir accès à un tel outil, qui aurait pu me donner les moyens d'atteindre mes objectifs ou même de les dépasser. Ce sujet me semble naturel, car l'outil lie mes deux passions, qui sont la natation et la technologie.

\vspace{12pt}

La problématique de mon travail est : \textbf{la technologie peut-elle remplacer le regard humain (en l'occurrence de l'entraineur) dans la technicité de l'entrainement ?} En soit, la technologie est utile pour seconder l'œil expert du coach, mais peut-elle finir par le remplacer complètement. Nous pourrons également voir si l'outil étudié permet un meilleur suivi que celui de l'entraineur.

\vspace{12pt}

Afin de répondre au mieux à ces questions, je vais dans un premier temps, définir les facteurs de performance en natation, que ce soient les différents types d'entrainement, l'alimentation et la santé, le mental ou encore la récupération.

Dans un second temps, je vais m'intéresser à l'outil, soit l'application en elle-même et aux données qu'elle fournit.

Dans un troisième temps, j'analyserai les données à disposition pour définir les apports et les limites de l'outil.

A partir de cette analyse, j'aurai toutes les clés en main pour proposer des axes d'amélioration de l'outil.

Enfin, je pourrai conclure et essayer de répondre au mieux au problème posé.

\vspace{12pt}

Dans le cadre de mon étude, je serai en contact avec l'équipe encadrant les nageurs de haut niveau du Cercle Paul Bert de Rennes. L'entraineur principal, monsieur Xavier Idoux a commencé à utiliser cet outil au début du mois de septembre 2016, et il est d'accord pour me transmettre des données (dans la mesure du possible). Il me sera également possible d'interagir avec un ostéopathe, monsieur Benjamin Degryse, qui utilise cet outil dans l'encadrement de divers sportifs.

Cet accord avec Xavier et monsieur Degryse me permettra de recueillir des données, le ressenti des nageurs, et aussi celui des entraineurs. 

  \clearpage
  \section{Comment performer en natation ?}

Avant de me demander si les nouvelles technologies peuvent finir par remplacer les entraineurs, je vais d'abord m'intéresser aux moyens qu'ils mettent en place pour que leurs nageurs performent. Premièrement, je vais exposer les différents types de séances d'entrainement, puis nous verrons que l'alimentation, la santé et la récupération ne sont pas à négliger, bien au contraire. Pour finir, nous verrons que le mental et la psychologie (relation entraineur/entrainé) se révèlent également très importantes.

Il est à noter que ce développement se basera sur des nageurs de haut niveau et non des nageurs occasionnels.

\subsection{Les différentes thématiques de séances}

La natation est un sport très ingrat qui demande beaucoup de sacrifices, pour souvent peu de résultat. Un nageur de haut niveau va devoir effectuer de nombreux entrainements, à raison de deux à trois par jour, tous les jours. Attention, ces chiffres sont variables en fonctions des nageurs, mais aussi des entraineurs.

\vspace{12pt}

En effet, un nageur devra plus ou moins nager qu'un autre, en fonction de \textbf{sa spécialité} : la nage et la distance sur laquelle il est le meilleur, mais aussi de \textbf{sa physiologie} : chaque être humain n'a pas les mêmes besoins ou qualités et c'est aussi valable à l'échelle du sportif. De plus, les entraineurs n'ont pas tous la même façon d'entrainer, certains vont privilégier \textbf{la quantité}, alors que d'autres \textbf{la qualité}.

\vspace{12pt}

Malgré ces différences, on remarque des similarités dans les types de séances. Il est possible de les regrouper en thématiques :
\begin{itemize}
 \item Endurance et séries
 \item Vitesse
 \item Spécifique, affutage et allures
 \item Technique et matériel
 \item Hors de l’eau
\end{itemize}

Pendant la semaine d'entrainement d'un nageur, ces cinq thématiques de séances seront forcément toutes abordées. Je vais vous expliquer plus en détails en quoi elles consistent.

\subsubsection{Endurance et séries}

Lors d'une séance d'endurance, l'accent est mis sur la quantité, c'est-à-dire que le nageur va nager longtemps : c'est un "gros" entrainement. Ces séances sont faites pour que le nageur garde le plus longtemps possible la bonne technique de nage et qu'il apprenne à adopter une technique de nage économe en énergie : l'amplitude. Nager en amplitude, c'est tenter de minimiser son nombre de coups de bras par longueur, en agrandissant son mouvement sans perdre l'efficacité de celui-ci.

\vspace{12pt}

D'un point de vue physiologique, ces entrainements permettent de travailler au niveau :

\begin{itemize}
\item \textbf{cardio-vasculaire}, le cœur va chercher à apporter plus rapidement l'oxygène aux organes
\item \textbf{musculaire}, le corps va tenter de produire de l'énergie à partir des lipides et des glucides plus facilement
\end{itemize}

\vspace{12pt}

Ce type de travail permet de conditionner les processus de récupération du nageur, ce qui lui permettra de maintenir une intensité élevée de plus en plus longtemps et efficacement.

\vspace{12pt}

J'ai dit qu'une séance d'endurance consiste à nager longtemps, mais cela ne signifie pas forcément de nager 8km sans s'arrêter. En effet, lors de ces entrainements, le nageur devra soit réaliser une très longue distance (la plupart du temps 2 ou 5km) ou plus souvent des \textbf{séries}. Une série est un élément indispensable de l'entrainement, c'est un enchainement de distances (identiques ou non) entrecoupé de temps courts de récupération.

En séance d'endurance, on retrouve souvent des séries telles que 20 x 100m ou encore 10 x 400m. Ou alors des pyramides, on commence à une petite distance, puis on augmente de plus en plus, pour redescendre à la première distance, par exemple : 100m, 200m, 400m, 200m puis 100m. Ou bien des séries descendantes, on part d'une grande distance que l'on réalise ensuite de façon fractionnée, par exemple : 800m, 2 x 400m, 4 x 200m, 8 x 100m, puis 16 x 50m.

\vspace{12pt}

Lors de ces séries, le temps de récupération du nageur est souvent déterminé par la vitesse à laquelle il réalise la distance. Par exemple, pour une série de 10 x 100m, l'entraineur peut choisir que le nageur "doit tenir un départ de 1min30", c'est-à-dire qu'il a 1min30 avant de devoir commencer le 100m suivant. Ainsi, s'il réalise le 100m en 1min10, il aura 20sec de récupération (ce qui est très bien pour une séance endurance).

On peut aussi choisir de fixer un temps de récup par distance (de 10 à 30 secondes).

\vspace{12pt}

Généralement, ce type de séances n'est pas très apprécié des nageurs, car ils nagent longtemps et ne peuvent pas se parler entre eux. Passons maintenant aux séances complètement à l'opposé des séances d'endurances : les séances de vitesse.


\subsubsection{Vitesse}\label{vit}

Sur ce type de séances, l'accent est mis sur la qualité, le nageur ne nagera pas longtemps (ce qui ne veut pas dire que l'entrainement ne durera pas longtemps), mais devra nager à une très haute intensité. Les séances de vitesse sont faites pour développer la vitesse maximale du nageur, mais aussi l'endurance de vitesse : garder sa vitesse maximale de plus en plus longtemps.

\vspace{12pt}

Si le nageur veut développer sa vitesse "pure", il devra réaliser des sprints, sur des distances allant de 10 à 25m, entrecoupés de temps de repos plutôt long (environ 2min). Il est nécessaire d'inclure des temps de récupération importants entre les sprints pour que le sportif puisse conserver une intensité maximale. Attention, ces temps de récupération sont le plus souvent effectués en récupération active (nage lente), ce qui permet une meilleure et plus rapide oxygénation des muscles. Cette récupération active est conseillée, car ces exercices se font essentiellement sans oxygène et créent un essoufflement.

Il est possible de faire des sprints uniquement sur les jambes (avec une planche et/ou des palmes), sur les bras (avec un pull entre les cuisses ou des plaquettes aux mains), ou encore en nage complète (avec ou sans matériel). Cela permet de mieux cibler les zones que l'on veut travailler.

\vspace{12pt}

De plus, pour travailler la vitesse pure, on peut aussi utiliser des élastiques de sur-vitesse. C'est un élastique attaché à une ceinture, puis tenu par une personne à l'autre bout. Le nageur avec la ceinture va se positionner le plus loin possible de la personne qui tient l'autre bout, en fonction des élastique on arrive à aller jusqu'aux 20m environ. Puis, la personne la commencer à tirer l'élastique tandis que le nageur va sauter dans l'eau et nager vers le mur en sprint, le plus vite et le mieux possible.

Cette technique de sur-vitesse permet de s'entrainer à garder de bons appuis même à une vitesse très élevée. Elle est souvent réalisée en période d'affutage (cette période sera expliquée dans la section \ref{spe} spécifique, affutage, allures).

\vspace{12pt}

Pour développer l'endurance de vitesse, il y a deux possibilités, soit on travaille avec \textbf{une récupération courte} : on diminue les temps de récupération (active ou non), soit on \textbf{augmente la distance} des sprints : jusqu'à 200m.

\vspace{12pt}

Les séries de sprints sont souvent une répétition de 50 ou 100m avec des sprints placés au début et le reste de la distance à faire en "souple" (nage facile), avec des temps de récupération (passive) d'environ 30sec.

\vspace{12pt}

Ce type de travail engendre une grande fatigue chez le nageur. En effet, ces séances sont très gourmandes en énergie et il est important de bien récupérer après celles-ci (l'importance de la récupération sera montrée dans la partie \ref{la recup} La récupération).
Malgré cela, ces séances sont souvent plus appréciées des nageurs, car elles sont généralement plus courtes et entrecoupées de temps de récupération importants. Ces temps de pause permettant aux nageurs de discuter entre eux, au plus grand désarroi des entraineurs. Effectivement, une séance de vitesse réalisée sans concentration ou investissement complet de la part du sportif, est moins efficace.

Voyons maintenant les séances spécifiques qui sont plus ou moins liées aux séances d'endurance et de vitesse.


\subsubsection{Spécifique, affutage et allures}\label{spe}

Comme son nom le laisse penser, cette thématique va se focaliser sur la spécialité du nageur. Comme nous avons pu le voir plus tôt, chaque nageur à sa nage et sa distance de prédilection, et c'est lors des séances spé qu'ils vont se concentrer dessus. L'entraineur va former des sous-groupes, il va regrouper \textbf{les fondeurs} (ou demi-fondeurs) : ceux qui nagent du crawl en longue distance (400m et plus) et ceux qui font du 400m 4n, puis \textbf{les sprinteurs} : ceux qui nagent des 50 ou 100m, pour finir avec \textbf{les intermédiaires} : ceux qui nagent les 200m.

Comme chaque personne a ses propres besoins, ces groupes ne sont pas figés. Le rôle du coach est de savoir quand un nageur doit être dans tel ou tel groupe.

\vspace{12pt}

Après avoir réparti les nageurs, l'entraineur va attribuer à chacun des trois groupes une séance spécifique, pour le groupe :

\begin{itemize}
\item sprint : ce sera du travail de vitesse. Souvent, on commence par une série pour travailler l'endurance de vitesse, puis on fini par travailler le sprint pur.
\item intermédiaire : en fonction de la programmation de l'entraineur, on aura des séries plus ou moins longues, pour travailler l'endurance, puis on finira par un peu de sprint
\item demi-fond : là ce sera des séries d'endurance, beaucoup plus longues que les autres groupes. Le kilométrage et la durée de l'entrainement pour ce groupe sera supérieur aux deux autres groupes. Puis, pour finir l'entrainement, comme le groupe intermédiaire, on aura un peu de sprint.
\end{itemize}

\vspace{12pt}

Le fait de travailler le sprint en dernier, permet de s'entrainer à \textit{relancer} à la fin de la course. Relancer signifie que le nageur va essayer d'accélérer dans les derniers mètres. C'est très important, surtout pour les distances de 400m ou plus, car si le nageur ne relance pas, il peut se faire dépasser et perdre \textit{à la touche}.

\vspace{12pt}

Exemple typique, lors d'un 400m, imaginons deux nageurs qui font la course ensemble. Dans ces cas là, il y a toujours l'un des deux qui dirige la course, qui donne le rythme et l'autre qui suit. Puis, aux 300 ou 350m si jamais celui qui suit, relance, et que l'autre n'est pas capable de sortir de son rythme, il va toucher derrière l'autre. 

Ainsi, bien que ce soit lui qui ai géré toute la course, il se sera fait battre dans les derniers mètres parce qu'il n'a pas su relancer.

\vspace{12pt}

Généralement, le fait de diviser le groupe d'entrainement par groupes spécifiques est lié à une période de préparation (des entrainements intensifs, en stage pendant les vacances scolaires) ou alors en période \textbf{d'affutage}.

L'affutage est la période précédant une compétition aux enjeux importants (qualifications ou championnats). Lors de ces périodes, l'entraineur va avoir recourt à des groupes spécifiques, mais les séances ne seront pas les mêmes que lors des périodes de préparation. En effet, elles seront de plus en plus courtes au plus l'échéance se rapprochera.

Les groupes feront de moins en moins de kilomètres par entrainement et feront de plus en plus :
\begin{itemize}
\item de travail de sprint pour les sprinteurs
\item de travail d'allure pour les deux autres groupes
\end{itemize}

\vspace{12pt}

On dit qu'un nageur doit faire des allures quand il doit réaliser des séries avec des temps les plus réguliers possibles. En plus d'être réguliers, les temps doivent correspondre au temps final visé pour la course.

Par exemple, si une nageuse veut faire 9min un 800m, elle devra viser des allures à 1'7 (1 minute et 7 secondes) au 100m (2'14 au 200m, 4'28 au 400m, et 8'56 au 800m). En réalisant régulièrement des 100m à ce temps précis, la nageuse habituera son corps à cette vitesse et sera capable de le reproduire en compétition.

\vspace{12pt}

Il existe de nombreux facteurs susceptibles d'influencer la performance lors du jour J, notamment la façon de nager. C'est pourquoi, les nageurs doivent impérativement effectuer un travail de technique. Ce travail permet à leur nage d'être le plus efficace possible peut importe la vitesse à laquelle elle est réalisée.


\subsubsection{Technique et matériel}

La technique se travaille par une répétition de différents exercices, mais aussi par l'utilisation de l'\gls{filet}. Les nageurs ont dans leur \gls{filet} :

\begin{itemize}
\item un tuba frontal
\item une planche
\item des palmes
\item un pull
\item un élastique (différent de l'élastique de sur-vitesse vu dans la section \ref{vit} vitesse)
\item des plaquettes
\item des petites plaquettes
\end{itemize}

Ces différents objets permettent de travailler un groupe musculaire en particulier, avec des variantes. Ils peuvent être utilisés dans n'importe quel type de séance, pour s'échauffer, récupérer ou se concentrer sur un groupe musculaire. On peut donc les regrouper en fonction des muscles qu'ils vont viser :
\begin{itemize}
\item le pull, l'élastique, les petites et grandes plaquettes pour \textbf{les bras}
\item la planche et les palmes pour \textbf{les jambes}
\end{itemize}
Le tuba frontal, quant à lui, est utilisé aussi bien pour solliciter les jambes que les bras. En effet, si on l'associe avec les palmes on travaillera les jambes, et si on l'utilise avec le pull, les plaquettes et l'élastique, on se concentrera sur les bras, sans se soucier de la tête qui sera fixe.

\vspace{12pt}

L'élastique, est un élastique en tissu ou en cuir, que l'on place au niveau des chevilles. On peut l'utiliser avec un pull pour plus de facilités. Il permet d'empêcher le nageur de faire des battements, afin de travailler les bras, mais aussi le gainage abdominal. Car, ce sont les jambes qui nous stabilisent dans l'eau.

\vspace{12pt}

Enfin, il existe plusieurs sortes de plaquettes, les plus utilisées sont les grandes (environ de la taille de la main ou un peu plus grandes) et aussi les petites qui ne recouvrent que les doigts.

Les petites vont permettre aux avant-bras de plus travailler, tandis que les grandes vont solliciter les bras entiers.

\vspace{12pt}

En plus du \gls{filet}, la technique se travaille par différents exercices. Ces exercices sont données aux personnes pour apprendre à nager ou pour améliorer leur nage, et aux nageurs pour garder une nage la plus efficace et moins gourmande en énergie possible. Car, à force de répéter les mouvements ou avec la fatigue, la technique du nageur peut se dégrader. De plus, il faut savoir que l'exigence par rapport à la technique de nage augmentera avec le niveau du nageur. En effet, la progression sera de plus en plus subtile, allant de quelques centièmes de secondes à quelques secondes par course, en fonction de la durée de celle-ci.

Ainsi, le nageur devra faire attention à la façon dont il rentre sa main dans l'eau, place son appuis sous l'eau, oriente sa poussée sous l'eau, tourne la tête pour respirer, etc...

Il faut également qu'il fasse attention au nombre de coups de bras effectués par 50m. Pour se fatiguer le moins possible, le nageur doit faire peu de coups de bras efficaces et respirer le moins possible, car le mouvement de la tête crée une résistance dans l'eau qui ralenti le nageur.

\vspace{12pt}

L'hypoxie est très utile pour habituer le nageur à minimiser son nombre de respiration par 50m. Elle consiste à réaliser une distance en définissant le nombre de coups de bras entre chaque respiration. Par exemple "400m respi 3,5,7" signifie que le nageur devra faire 3 coups de bras puis respirer, puis 5 coups de bras, puis 7, puis recommencer, le tout pendant 400m. 

\vspace{12pt}

Le travail de la technique de nage est très important, mais il doit s'accompagner d'un travail sur \gls{parties non nagees}, qui sont :
\begin{itemize}
\item le \gls{depart}
\item le \gls{coulees}
\item le \gls{virages}
\end{itemize}

\vspace{12pt}

Le travail du \gls{depart} permet de diminuer le temps de réaction et de perfectionner l'entrée dans l'eau du nageur, il va souvent de pair avec le travail des \gls{virages}. Aussi bien pour le \gls{depart} que les \gls{virages}, le nageur doit pousser fort sur ses jambes, afin de nager le moins possible.

Enfin, pour travailler les \gls{coulees}, le nageur devra faire des séries d'apnée et s'imposer de faire des coulées longues (au moins 7m) tout au long de l'entrainement. Une série d'apnée est souvent constituée de plusieurs 50m réalisés en apnée et séparés par de la récupération passive.

La dernière façon de maximiser sa progression en natation et de coupler tout ce que l'on vient de voir avec un travail en dehors de l'eau.


\subsubsection{Hors de l'eau}

Le fait de faire des exercices en dehors de l'eau permet de développer plus facilement la force, la souplesse et la réactivité du nageur. Mais cela permet également de travailler les groupes musculaires moins utilisés dans l'eau, pour éviter les blessures dues à des déséquilibres musculaires.

Avant d'aller dans l'eau, le nageur va s'échauffer \gls{a sec} ou effectuera une séance de muscu ou ppg. Après l'entrainement \gls{a sec} et l'entrainement dans l'eau, il devra finir par des étirements, pour aider ses muscles à récupérer. Certains étirements permettent de réveiller en douceur les muscles, c'est ceux là qu'on fera avant l'entrainement. Tandis que d'autres permettent de gagner de la souplesse, de prévenir les blessures et de récupérer, ceux-ci se font à la fin de l'entrainement. Il est important de ne pas négliger les étirements, car un nageur souple est un nageur capable de faire des mouvements plus amples et donc plus efficaces.

\vspace{12pt}

Lors d'un échauffement, le nageur va donc pouvoir effectuer des étirements, mais aussi certains exercices de ppg pour préparer ses muscles. Des abdos, des pompes, du gainage, etc... Le \gls{proprioception} est également indispensable. Il se réalise avec ou sans matériel et permet, comme les étirements, d'éviter bien des blessures.

\vspace{12pt}

Enfin, le travail de renforcement musculaire des nageurs passe par des séances de ppg et de muscu. La ppg est souvent une séance de circuit training réalisée au poids du corps et en enchainant différents exercices. Pendant une séance de ppg, le nageur peut utiliser divers objets, tels qu'une corde à sauter, un TRX, un SwissBall, une corde, etc...

\begin{figure}[h]
\begin{center}
 \includegraphics[scale=0.5]{img/TRX.png}
 \end{center}
 \caption{Exemple d'exercice de gainage avec un TRX}
 \label{Exemple d'exercice de gainage avec un TRX}
\end{figure}

\begin{figure}[h]
\begin{center}
 \includegraphics[scale=0.47]{img/SwissBall.png}
 \end{center}
 \caption{Exemple d'exercice de gainage avec un SwissBall}
 \label{Exemple d'exercice de gainage avec un SwissBall}
\end{figure}

La muscu est réalisée à l'aide de machines, barres ou poids, afin de rendre plus dur les exercices et donc d'augmenter la masse musculaire. Vous trouverez en Annexe \ref{Annexe A} un exemple de séance de muscu du CPB de Rennes, fourni par le préparateur physique, Pierre Chenaut.

\vspace{12pt}

L'ensemble des thématiques étudiées montre globalement les entrainements qu'effectuent les nageurs pour progresser. Cependant, l'entrainement n'est pas l'unique facteur qui influence la performance en natation, c'est pourquoi nous allons maintenant nous intéresser à l'alimentation, mais aussi la santé.


\subsection{L’alimentation et la santé}

L'alimentation d'un nageur est très différente de celle d'une personne sédentaire ou ayant une activité normale, du fait de sa dépense énergétique importante. Ainsi, un nageur doit combler ses dépenses par la nourriture, mais aussi par l'hydratation.

\vspace{12pt}

Dans l'eau on a tendance à ne pas ressentir la soif et penser qu'il est inutile de s'hydrater. Cependant, même si l'on ne transpire presque pas dans l'eau (le corps se refroidissant grâce à l'\gls{conduction}), on se dépense et on a besoin de bien s'hydrater. En effet, la déshydratation peut provoquer des blessures, limiter la récupération et diminuer les performances.

Les nageurs doivent donc s'hydrater aussi bien pendant la journée, que pendant leurs entrainements.

\vspace{12pt}

Après l'hydratation, nous allons nous intéresser aux dépenses énergétiques et aux solutions pour y pallier. Le besoin calorique d'un adulte ayant une activité moyenne est d'environ 2500kcal, tandis que celui d'un nageur est bien plus élevé et dépend du nombre, de la durée et de l'intensité de ses séances d'entrainements. Pour un nageur de haut-niveau, on peut aller jusqu'à une dépense de 1000kcal par heure d'entrainement. D'ailleurs, l'un des exemple le plus parlant serait selon moi celui de Michael Phelps, le meilleur nageur du monde, le plus titré et avec le plus grand nombre de record du monde. On trouve facilement sur internet le fait que Phelps aurait un apport journalier de 12000kcal.

\vspace{12pt}

Bien qu'il ai démenti ce chiffre, cela ne semble pas impossible, étant donné qu'il lui arrive de s'entrainer plus de 8h par jour (dont au moins 6h dans l'eau). Donc, si on additionne l'apport journalier normal aux 1000kcal par heure d'entrainement, cela donne un total de plus ou moins 10500kcal par jour. Il faut évidemment relativiser, Phelps est un nageur de niveau olympique, et il ne réalise pas 8h d'entrainement tout au long de l'année, il y a des périodes d'entrainements plus ou moins intensives comme on a plus le voir précédemment.

Plus précisément, lors d'une période d'affutage (voir section \ref{spe} spécifique, affutage, allures) le régime alimentaire du nageur va diminuer avec le kilométrage. Enfin, la veille de la compétition, le nageur devra faire attention à ne pas manger des aliments différents ou qu'il digère mal pour faciliter sa digestion.

\begin{figure}[h]
\begin{center}
 \includegraphics[scale=0.56]{img/CalculCalories.png}
 \end{center}
 \caption{Outil de natationpourtous.com pour calculer les calories}
 \label{Outil de natationpourtous.com pour calculer les calories}
\end{figure}

Le site internet natationpourtous.com met à disposition un outil pour calculer le nombre de calories brûlées en nageant en fonction de la personne (taille et poids), mais aussi des nages. Cela laisse penser qu'en fonction des thématiques d'entrainement, la dépense énergétique peut également varier. Ainsi, en fonction de plusieurs facteurs, le nageur devra adapter son régime alimentaire, mais dans tous les cas, on remarque qu'il est bien supérieur à la normale. Pour pallier à cette dépense, le nageur aura un régime spécial, mais aura également recourt à des compléments alimentaires.

\vspace{12pt}

Encore une fois, il est nécessaire pour le nageur de combler ses dépenses énergétiques, sinon il risque de se blesser ou de tomber malade plus facilement. En effet, l'alimentation est étroitement liée à la santé du sportif, car s'il a des carences, il sera plus fragile. Or un nageur malade est un nageur qui ne s'entraine pas ou pas totalement, et dont les performances diminuent, c'est donc à éviter.

\vspace{12pt}

Nous avons déjà à montré l'importance de la récupération, maintenant voyons les moyens mis en place pour l'optimiser.


\subsection{La récupération}\label{la recup}

Étirements, nourriture, hydratation, sommeil (à développer), récupération active, bas de contention, massages (kiné et ostéo)... Conséquences de la négligence de ca
Image de la quantité idéale de sommeil (comparatif sportif/normal)

Éviter le surentrainement : http://www.natationpourtous.com/entrainement/surentrainement.php

\vspace{12pt}

Pour finir, nous allons voir que même si toutes les conditions sont réunies pour que le nageur améliore ses performances, il se peut que ce ne soit finalement pas le cas, car psychologiquement il n'était pas prêt le jour J.


\subsection{Le mental et la psychologie}

Todo

Essayer de trouver des chiffres pour l'importance du mental
Le stress : http://www.natationpourtous.com/entrainement/motivation.php
La relation entraineur/entrainé
Conséquences de la négligence de ca

On a vu la performance en natation, mais c'est valable pour tous les sports

\clearpage
\section{Qu’est-ce que totalcoaching.com ?}
\subsection{L’application en détails}

Todo


Elle permet de gérer les différents paramètres menant à la performance :
Alimentation
Humeur
Fonctionnalité (souplesse, musculation, proprioception, ...)

Contact avec le créateur, mais refus de sa part de répondre aux questions.

Recueillir des témoignages d'utilisateurs

Adaptable à bien d'autres sports


\subsection{Les données recueillies}

Todo

Voir avec Xavier pour les données


\clearpage
\section{Analyse de l’outil}

Est-il plus efficace que l’outil "entraineur" ?

Voir si autre appli du même genre

\subsection{Le monitoring des données des nageurs de Rennes}
\subsubsection{Au quotidien}

Todo


\subsubsection{A l’entrainement}

Todo


\subsubsection{En compétition}

Todo


\subsection{Les apports et limites}

Todo


Cette application ne serait-elle pas plutôt le meilleur adjoint de l'entraineur ?


\subsection{Les autres outils}

Todo

Le générateur d'entrainement
http://www.natationpourtous.com/entrainement/generateur/generateur.php


\clearpage
\section{Mon apport possible}

Todo

Ce qui manque à l'outil et que je serais susceptible d'ajouter

  \clearpage
  \section{Conclusion}

\subsection{Perspectives et bilan}

\subsection{Problèmes et difficultés résolus}

\subsection{Conclusion}

  \clearpage
  \bibliographystyle{unsrt}
  \nocite{*}							% Pour afficher toutes les entrées de la biblio (même sans référence)
  \bibliography{bibliographie}
  \clearpage
  \appendix
  \setcounter{page}{1}				% Pour recommencer la numérotation (annexes)
  \pagenumbering{Roman}				% Pour que la numérotation soit en numéros romains
  \chapter{Exemple d'une séance de muscu}\label{Annexe A}


  \clearpage
  %%%%%%%%%%%%%%%%%%%%%%%%%%% Acronymes %%%%%%%%%%%%%%%%%%%%%%%%%%%

\newacronym{filet}{filet}{ensemble des affaires d'entrainement d'un nageur}

\newacronym{parties non nagees}{parties non nagées}{l'ensemble des moments de la course où le nageur ne nage pas vraiment}

\newacronym{virages}{virages}{moment où le nageur fait demi-tour au niveau du mur}

\newacronym{depart}{départ}{moment où le nageur effectue son entrée dans l'eau}

\newacronym{coulees}{coulées}{moment où le nageur est sous l'eau, après un départ ou un virage}

\newacronym{a sec}{à sec}{en dehors de l'eau}

\newacronym{proprioception}{proprioception}{renforcement des muscles et réflexes qui nous permettent de garder l'équilibre}

\newacronym{conduction}{conduction}{action du corps pour réguler sa température au contact de l'eau}

\newacronym{surentrainement}{surentrainement}{période où le nageur est trop fatigué}

\newacronym{sophrologie}{sophrologie}{techniques pour être serein}


  \glsaddall							% Permet de dire que l'on veut afficher tous les acronymes, même ceux sans lien
  \printglossaries					% Afficher le glossaire
  \addcontentsline{toc}{chapter}{Glossaire}	% Ajout dans la table des matières
  \clearpage	
  \setcounter{tocdepth}{5} 			% Profondeur du sommaire
  \tableofcontents					% La table des matières (profondeur maximale)
  \addcontentsline{toc}{chapter}{Table des matières}	% Ajout dans la  table des matières
  \clearpage
  \listoffigures						% Table des figures
  \addcontentsline{toc}{chapter}{Table des figures}	% Ajout dans la table des matières
  \printnomenclature
  \addcontentsline{toc}{chapter}{Table des sigles et abréviations}	% Ajout dans la table des matières
  \clearpage
\end{document}