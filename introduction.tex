\section*{Introduction}
\addcontentsline{toc}{section}{Introduction}	% Ajout dans la table des matières


Dans le cadre de ma deuxième année de master MIAGE en apprentissage, j'ai pu intégrer l'entreprise AXA France, pour une durée d'un an, afin de compléter au mieux ma formation scolaire. Au cours de cette année, j'ai intégré le service "prévention risques activités opérationnelles et gestion de crise" au sein de la direction de la sureté, elle-même au sein du secrétariat général. Ma mission a pour descriptif "assistante chef de projet informatique" et consiste à prévenir/identifier des fraudes internes grâce à l'outil ACL, mais aussi participer à la formation des collaborateurs à l'outil ACL, animer des réunions, ect...

Lors de cette année d'apprentissage j'ai et je continue d'apprendre énormément, aussi bien à l'université que dans l'entreprise AXA France grâce à ma tutrice {\MaitreApp}. Cependant, suite aux conseils de mes enseignants, j'ai choisi de réaliser mon mémoire sur un sujet complètement différent de ce sur quoi j'ai pu travailler en entreprise.

\vspace{12pt}

Ce choix, s'est naturellement porté sur un sujet concernant l'entrainement sportif et même plus précisément l'entrainement en natation. J'ai réussi, depuis le collège, à concilier mes études avec ma pratique de la natation à haut niveau. Ma passion pour ce sport explique mon choix.

Comme mes études ne concernent en rien la natation, je vais m'intéresser à un outil utilisable dans le quotidien d'un nageur. Cet outil, totalcoaching.com, est une application multi-device qui peut être utilisé par des nageurs, mais aussi par n'importe quel autre sportif. Il permet de gérer les différents paramètres d'accès à la performance, que j'expliciterai au cours de ce mémoire, mais aussi d'éviter le surentrainement.

Lors de ma pratique sportive, j'aurai aimé avoir accès à un tel outil, qui aurait pu me donner les moyens d'atteindre mes objectifs ou même de les dépasser. Ce sujet me semble naturel, car l'outil lie mes deux passions, qui sont la natation et la technologie.

\vspace{12pt}

La problématique de mon travail est : \textbf{la technologie peut-elle remplacer le regard humain (en l'occurrence de l'entraineur) dans la technicité de l'entrainement ?} En soit, la technologie est utile pour seconder l'œil expert du coach, mais peut-elle finir par le remplacer complètement. Nous pourrons également voir si l'outil étudié permet un meilleur suivi que celui de l'entraineur.

\vspace{12pt}

Afin de répondre au mieux à ces questions, je vais dans un premier temps, définir les facteurs de performance en natation, que ce soient les différents types d'entrainement, l'alimentation et la santé, le mental ou encore la récupération.

Dans un second temps, je vais m'intéresser à l'outil, soit l'application en elle-même et aux données qu'elle fournit.

Dans un troisième temps, j'analyserai les données à disposition pour définir les apports et les limites de l'outil.

A partir de cette analyse, j'aurai toutes les clés en main pour proposer des axes d'amélioration de l'outil.

Enfin, je pourrai conclure et essayer de répondre au mieux au problème posé.

\vspace{12pt}

Dans le cadre de mon étude, je serai en contact avec l'équipe encadrant les nageurs de haut niveau du Cercle Paul Bert de Rennes. L'entraineur principal, monsieur Xavier Idoux a commencé à utiliser cet outil au début du mois de septembre 2016, et il est d'accord pour me transmettre des données (dans la mesure du possible). Il me sera également possible d'interagir avec un ostéopathe, monsieur Benjamin Degryse, qui utilise cet outil dans l'encadrement de divers sportifs.

Cet accord avec Xavier et monsieur Degryse me permettra de recueillir des données, le ressenti des nageurs, et aussi celui des entraineurs. 
